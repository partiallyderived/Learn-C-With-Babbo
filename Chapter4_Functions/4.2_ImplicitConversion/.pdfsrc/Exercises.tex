\documentclass{article}
\usepackage{inconsolata}
\usepackage{indentfirst}
\usepackage{listings}
\usepackage{xcolor}
\usepackage{textcomp}
\usepackage[T1]{fontenc}

\definecolor{cbackground}{rgb}{0.9, 0.9, 0.9}
\definecolor{cstring}{rgb}{0, 0.5, 0}

\lstset{language=C, upquote=true}
\lstdefinestyle{CStyle}{
    backgroundcolor=\color{cbackground},
    keywordstyle=\color{blue},
    stringstyle=\color{cstring},
    basicstyle=\ttfamily,
    breakatwhitespace=false,         
    breaklines=true,                 
    captionpos=b,                    
    keepspaces=true,              
    numbersep=5pt,                  
    showspaces=false,                
    showstringspaces=false,
    showtabs=false,                  
    tabsize=2,
    language=C
}


\begin{document}
\title{Exercises 4.2}
\author{Learn C With Babbo}
\date{}
\maketitle

\begin{enumerate}
\item Give the output of the following statements

\begin{enumerate}
\item 
\begin{lstlisting}[style=CStyle]
int a = 1.5;
printf("%d\n", a);
\end{lstlisting}

\item
\begin{lstlisting}[style=CStyle]
float a = 3 + 9;
printf("%.0f\n", a);
\end{lstlisting}

\begin{lstlisting}[style=CStyle]
int a = 3.14;
float b = a * 2;
printf("%.2f\n", b);
\end{lstlisting}
\end{enumerate}

\item For each of the following items, say what the type of the result of the expression is.
\begin{enumerate}

\item 
\begin{lstlisting}[style=CStyle]
3 * 7
\end{lstlisting}

\item
\begin{lstlisting}[style=CStyle]
2 / 7.4
\end{lstlisting}

\item 
\begin{lstlisting}[style=CStyle]
2L + 3.7f
\end{lstlisting}

\item
\begin{lstlisting}[style=CStyle]
3 % 7L
\end{lstlisting}

\item
\begin{lstlisting}[style=CStyle]
2.5f + 3.5
\end{lstlisting}

\item
\begin{lstlisting}[style=CStyle]
2.5 + 3.5
\end{lstlisting}
\end{enumerate}

\item Suppose I have the following function defined:
\begin{lstlisting}[style=CStyle]
double combine(int num1, double num2, int num3) {
    return num1 + num2 + num3;
}
\end{lstlisting}

What is the output of the following statement?
\begin{lstlisting}[style=CStyle]
printf("%.2f\n", combine(6.4, 8, -1.5));
\end{lstlisting}

\item Explain why implicit conversion of integral values to floating point values is generally harmless, but the 
implicit conversion of floating point values to integral values is often undesirable.
\end{enumerate}
\end{document}