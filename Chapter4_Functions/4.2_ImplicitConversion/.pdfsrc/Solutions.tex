\documentclass{article}
\usepackage{inconsolata}
\usepackage{indentfirst}
\usepackage{listings}
\usepackage{xcolor}
\usepackage{textcomp}
\usepackage[T1]{fontenc}

\definecolor{cbackground}{rgb}{0.9, 0.9, 0.9}
\definecolor{cstring}{rgb}{0, 0.5, 0}

\lstset{language=C, upquote=true}
\lstdefinestyle{CStyle}{
    backgroundcolor=\color{cbackground},
    keywordstyle=\color{blue},
    stringstyle=\color{cstring},
    basicstyle=\ttfamily,
    breakatwhitespace=false,         
    breaklines=true,                 
    captionpos=b,                    
    keepspaces=true,              
    numbersep=5pt,                  
    showspaces=false,                
    showstringspaces=false,
    showtabs=false,                  
    tabsize=2,
    language=C
}


\begin{document}
\title{Solutions 4.2}
\author{Learn C With Babbo}
\date{}
\maketitle

\begin{enumerate}
\item

\begin{enumerate}
\item 
\begin{lstlisting}[style=CStyle]
1
\end{lstlisting}

\item
\begin{lstlisting}[style=CStyle]
12
\end{lstlisting}

\begin{lstlisting}[style=CStyle]
6.00
\end{lstlisting}
\end{enumerate}

\item
\begin{enumerate}

\item 
\begin{lstlisting}[style=CStyle]
int
\end{lstlisting}

\item
\begin{lstlisting}[style=CStyle]
double
\end{lstlisting}

\item 
\begin{lstlisting}[style=CStyle]
float
\end{lstlisting}

\item
\begin{lstlisting}[style=CStyle]
long
\end{lstlisting}

\item
\begin{lstlisting}[style=CStyle]
double
\end{lstlisting}

\item
\begin{lstlisting}[style=CStyle]
double
\end{lstlisting}
\end{enumerate}

\item \verb|12| (note that \verb|-1.5| is rounded down to \verb|-2|)

\item The conversions from integral values to floating point values are usually exact, while the conversion of a 
floating point value to an integral value will result in the loss of the fractional part of that value.
\end{enumerate}
\end{document}