\documentclass{article}
\usepackage{inconsolata}
\usepackage{indentfirst}
\usepackage{listings}
\usepackage{xcolor}
\usepackage{textcomp}
\usepackage[T1]{fontenc}

\definecolor{cbackground}{rgb}{0.9, 0.9, 0.9}
\definecolor{cstring}{rgb}{0, 0.5, 0}

\lstset{language=C, upquote=true}
\lstdefinestyle{CStyle}{
    backgroundcolor=\color{cbackground},
    keywordstyle=\color{blue},
    stringstyle=\color{cstring},
    basicstyle=\ttfamily,
    breakatwhitespace=false,         
    breaklines=true,                 
    captionpos=b,                    
    keepspaces=true,              
    numbersep=5pt,                  
    showspaces=false,                
    showstringspaces=false,
    showtabs=false,                  
    tabsize=2,
    language=C
}


\begin{document}
\title{Programming Assignment 5.3}
\author{Learn C With Babbo}
\date{}
\maketitle
First, think of your favorite color and your favorite number (come on, we all have one). You will prepare two
prompts for the user. The first should ask for the user's favorite color. If the user enters your favorite color,
you should make some kind of remark (e.g., ``Hey, that's my favorite color too!''). Do the same for the user's 
favorite number. Afterwards, conditionally print an additional remark based on whether either the user has entered both 
your favorite color and number (e.g., ``You've got good taste'') or if the user has entered neither your favorite 
color or number (e.g., ``Hmm, I guess we don't have much in common''). You may find it useful to refer back to 
Chapter 3 and lesson 2.4.
\end{document}