\documentclass{article}
\usepackage{graphicx}
\usepackage{indentfirst}

\begin{document}
\title{File Systems and Data}
\author{Learn C With Babbo}
\date{}
\maketitle
\section*{File Systems}
A \textbf{file system} is a system for the organization, storage, and retrieval of data on a computer. Almost
invariably you will find that modern filesystems consist of \textbf{files} and \textbf{directories} 
(or, less formally, \textbf{folders}). Files are, fundamentally, named pieces of data on a file system. Files have a 
\textbf{size} indicating the number of \textbf{bytes} they consist of, which I will elaborate on in a later section.
Directories, on the other hand, are components of your file system which have links to files and other directories.
Likely you already have a good idea about files and directories just from navigating your file system with your operating system's
file manager. On Windows, if you press the Windows key and type ``My Computer" or ``This PC"
and selected the first option, you'll see your C drive (unrelated to the programming language), which you can navigate
into and explore all the files and directories on your system. On Windows, this interface is called
\textbf{Windows Explorer}. Mac users may navigate their file system using the \textbf{Finder} app, and on Linux systems,
there are several graphical file system managers. As I said, you are probably already familiar with the file system
manager that you use, but I want to bring it up so that you understand that when I am talking about file systems
I am talking about the organization of files into directories that you find on modern computers.
\section*{Bits and Bytes}
A byte is the smallest amount of data that may be stored in memory. Bytes themselves consist of 8 \textbf{bits}. 
Abstractly speaking, a bit is an on-off switch: it is always in one of two states: A bit is either ``on" or ``off", 
or one might say a bit indicates ``yes" or ``no", or that a bit is either ``one" or ``zero." Bits are the smallest pieces 
of information available on a computer. A byte is the smallest chunk of data that may be
\textit{stored}, which is reflected in the fact that the smallest non-empty file that may exist on a file system has
a size of one byte, not one bit. All data on a computer, whether they are names of programs, programs themselves, numbers, 
words, or anything else, are
represented as series of bytes. The next three sections will describe how numbers, words, and programs are represented
this way.
\section*{Representation of Numbers}
We normally represent numbers is through the base 10 \textbf{decimal} system, whereby the 10 digits from 0 to 9 
are written in a sequence wherein they are multiplied by a power of ten according to their position in the sequence,
and then summed together to represent a number. For example, the number 7401 may be digitally decomposed as follows: \\ \\
$7401 = 7000 + 400 + 0 + 1 = 7 * 1000 + 4 * 100 + 0 * 10 + 1 * 1 = 7 * 10^3 + 4 * 10^2 + 0 * 10^1 + 1 * 10^0$ \\ \\
(Note that any number to the power of zero is 1).
The decimal system is called base 10 because numbers are decomposed into powers of 10. Indeed, numbers may be decomposed
in an almost identical way for any positive integer that is at least 2. A base 4 number system, for example, uses only
the digts 0, 1, 2, and 3. The number 99 in base four, for example, is represented as $1203_4$, where the subscript
4 is used to denote that this is a base 4 and not a base 10 representation. To see why, let's decompose $1203_4$ into its
base 4 digits: \\ \\
$1203_4 = 1 * 1000_4 + 2 * 100_4 + 0 * 10_4 + 3 * 1_4 = 1 * 4^3 + 2 * 4^2 + 0 * 4^1 + 3 * 4^0 = 
1 * 64 + 2 * 16 + 0 * 4 + 3 * 1 = 64 + 32 + 0 + 3 = 99$ \\ \\
Note carefully how $1000_4 = 4^3, 100_4 = 4^2, 10_4 = 4^1, 1_4 = 4^0$: this is completely analogous to how 
1000, 100, 10, and 1 are powers of 10 in the decimal system. \\

So why am I even talking about bases? Because bits give us a convenient mechanism for representing numbers in base 2,
commonly called \textbf{binary}. This is because every bit could be considered a base 2 digit, of which there are only
2: 0 and 1. The number 13 has the binary representation $1101_2$ as we may see by expanding $1101_2$ as \\ \\
$1101_2 = 1 * 2^3 + 1 * 2^2 + 0 * 2^1 + 1 * 2^0 = 1 * 8 + 1 * 4 + 0 * 2 + 1 * 1 = 8 + 4 + 0 + 1 = 13$ \\ \\
This is indeed how numbers are primarily represented in computers, with variations for negative numbers and fractional
(usually called ``floating-point") numbers. When decimal (again, base 10) numbers are displayed on your computer screen,
such as in a calculator app, your computer is converting bits, which may essentially be seen as base 2
digits, to base 10 digits, which itself would turn out to be a sequence of bytes where every byte represents one base
10 digit, more details of which we'll get into in the next section.
\section*{Representation of Text}
It turns out that the ability of bits and bytes to represent numbers also gives us an easy way to represent text. One
common way of doing so is through \textbf{ASCII}, which stands for American Standard Code for Information Interchange.
In this system, each number which may be represented in one byte has a symbolic counterpart, which may be a letter,
a textual representation of a digit, a symbol, or a space. Note that a byte, being composed of 8 bits, may represent
a number as large as 255, which is $11111111_2 = 2^8 - 1$. This is analogous to the fact that the largest number that may be 
represented with 8 decimal digits is $99999999 = 10^8 - 1$. The symbolic counterparts may be found in an 
\textbf{ASCII Table}, shown below: \\

\includegraphics{asciifull.png} \\

The two most relevant columns for us are ``Dec" and ``Chr", ``Dec" corresponding to the numeric value of the byte, while
``Chr" corresponds to the character representation. For example, note that 65 in the Dec column corresponds to ``A" in the
``Chr" column, so the same assortment of bits that represents the 65 may also represent the capital letter ``A". Also,
we can find that the number 53 corresponds to the digit ``5": this gives us a way to represent base 10 numbers in binary.
For example, to display the number ``7301", we may use the bytes with numeric values 55, 51, 48, and 49, which correspond
to the digits 7, 3, 0, and 1 respectively. Notice also that every symbol that may be typed with standard keyboards
is also in the ASCII table, including the exclamation point, plus symbol, asterisk, etc, along with a space character,
with value 32. Somewhere in any text editor you use, such as Microsoft Word, is a series of bytes corresponding to the
contents of that program, and standard characters will be stored according to the ASCII table.\\

I would like to stress that bits and bytes fundamentally represent neither numbers nor text: as I said before, bits
are fundamentally one of two states, and bytes are composed of 8 bits. How we interpret sequences of bits and
bytes ultimately depends on context.
\section*{Representation of Programs}
I'll conclude this lesson of how bits and bytes represent data by describing how they represent programs. Unlike
numbers and text, programs are made to be understood by computers, not by humans (note in this case I am not talking
about the source code used to create the program, but the program itself). In doing so, it uses bytes that represent
different instructions natively understood by your computer, creating an execution blueprint that your computer 
understands. Because of this, the content of programs is not readable by most humans, though dedicated individuals
could learn to read it in principle. Files not meant to be read by humans are usually called 
\textbf{binary files}, and include other kinds of files such as
compressed data. At the risk of sounding repetitive, \textit{all} data on a computer is composed of bytes.

\end{document}