\documentclass{article}
\usepackage{applekeys}
\usepackage{hyperref}
\hypersetup{
    colorlinks=true,
    urlcolor=blue
}
\usepackage{indentfirst}
\usepackage{keystroke}
\begin{document}
\title{Installing gcc}
\author{Learn C With Babbo}
\date{}
\maketitle
\section*{gcc}
Because C is a compiled language, we will need to install a C compiler for your system
if it does not already have one. The C compiler program is called \textbf{gcc}, which stands for
``GNU Compiler Collection". I will give
directions on how to install gcc for the various operating systems in the following sections.
\section*{Windows}
\begin{enumerate}
\item Go to \url{http://www.mingw.org}
\item Click the ``Download Installer" button. A download should start automatically.
\item Open the downloaded file, named ``mingw-get-setup.exe" at the time of writing.
\item Click ``Install", then click ``Continue"
\item Wait for installation to complete, then click ``Continue"
\item You should see an interface listing packages that may be installed. Check ``mingw32-gcc-g++" and then
click ``Mark for installation."
\item Click on ``Installation" in the top left and select ``Apply Changes."
\item Wait for installation to complete, and then close the dialog box.
\item Press the Windows key and type ``environment variables" and select the first option.
\item Click on the ``Environment Variables..." button at the bottom right.
\item In the top box, click on the entry with ``Path" under the variable column.
\item Click the ``Edit..." button
\item If you have an ``Add..." button, click it and paste (without quotes) ``C:\textbackslash MinGW\textbackslash bin" into the new entry.
If you don't have an add button, paste (without quotes, note the semi-colon) ``;C:\textbackslash MinGW\textbackslash bin" to the end of the
Path content, without modifying the existing Path.
\item Press ``OK" until you have closed the window with the ``Environment Variables..." button. \\
\textbf{Important}: Do not click ``Cancel" or close the windows with the X in the top right: this will discard your changes.
\item Open the Command Prompt (Windows Key + R, and then enter ``cmd"). Enter \verb|gcc|. You
should see something like \\ \\
\verb|gcc: fatal error: no input files| \\
\verb|compilation terminated.| \\ \\
If so, your installation was successful. If instead you see something like \\ \\
\verb|'gcc' is not recognized as an internal or external command,| \\
\verb|operable program or batch file.| \\ \\
then something went wrong in your installation. If the file \verb|C:\MinGW\bin\gcc| exists on your system,
you have probably not set your Path environment variable correctly. Also, you need to have opened a \textit{new}
Command Prompt after editing the Path environment variable, \verb|gcc| will not be recognized if you attempt to 
use an already opened Command Prompt.
\end{enumerate}

\section*{Mac}
\begin{enumerate}
\item Open the AppStore and search for Xcode.
\item Install Xcode.
\item After Xcode is done installing, open a terminal (\cmdkey + \Spacebar, enter ``Terminal") and enter gcc. 
You should see something like \\ \\
\verb|gcc: fatal error: no input files| \\
\verb|compilation terminated.| \\ \\
If so, your installation was successful. If instead you see something like \\ \\
\verb|gcc: command not found| \\ \\
then something went wrong in your installation. Make sure that installation has finished, and that you opened
the terminal \textit{after} installation was complete, and did not use a previously open terminal.
\end{enumerate}

\section*{Linux}
You probably already have \verb|gcc| installed. To check, open a terminal and enter \verb|gcc|. 
If you see something like \\ \\
\verb|gcc: fatal error: no input files| \\
\verb|compilation terminated.| \\ \\
You already have \verb|gcc|. Otherwise, you should install it using a package manager available for your
linux distribution. Try the following commands: \\ \\
\verb|sudo apt-get update && apt-get install gcc| \\
\verb|sudo yum install gcc| \\ \\
If neither of these commands work, enter \verb|cat /etc/*-release| to get your linux distribution name
(there should be a line similar to \verb|NAME=<name>|), and use
a search engine like Google with the query ``install gcc on \textless distribution name\textgreater" to get help on installing
\verb|gcc| on your distribution, or look up how to install it from source.

\end{document}