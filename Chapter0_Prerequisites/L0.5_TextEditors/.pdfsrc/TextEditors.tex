\documentclass{article}

\usepackage{hyperref}
\usepackage{indentfirst}

\hypersetup{
    colorlinks=true,
    linkcolor=blue,
    urlcolor=blue
}

\begin{document}
\title{Text Editor Recommendations}
\author{Learn C With Babbo}
\date{}
\maketitle

\section*{}
Before we start programming, it will help to have a good text editor which has a \textbf{syntax highlighting}
feature, which will use colors and formatting to distinguish between various components of your code. While it is
possible to code in programs like Notepad or even command line editors such as vim, editors designed for programmers 
will make your code appear less cluttered and are more visually appealing. That being said, if you want to use a plain
text editor such as Notepad, that's fine too. For the purposes of learning your first programming language, 
I would recommend against the use of an \textbf{Integrated Development Environment}, or IDE for short, 
because new programmers may often become overly reliant on their advanced features and fail to learn the proper language
semantics themselves. If you don't know what an IDE is, don't worry about it. The following sections will contain my 
text editor recommendations depending on your operating system.

\section*{Windows}
For windows, I recommend using \textbf{Notepad++}, which may be found 
\href{https://notepad-plus-plus.org/download/v7.5.6.html}{here}. If you aren't sure which download to take, just click
the big green ``Download" button and use that. After you've downloaded Notepad++, run the installer and follow its
instructions.

\section*{Mac}
For Mac, I recommend using TextWrangler. To find it, open the AppStore and search for it.

\section*{Linux}
For Linux, I recommend \textbf{gedit}. How to install it will depend on your Linux distribution. Here are some commands
in bash that will let you install gedit. Note that it may already be installed, which you may check by entering ``gedit"
in bash. \\ \\
Ubuntu: \verb|sudo apt-get install gedit| \\ \\
Red Hat: \verb|sudo yum install gedit|

\end{document}