\documentclass{article}

\usepackage{indentfirst}
\usepackage{hyperref}
\hypersetup{
    colorlinks=true,
    linkcolor=blue,
    urlcolor=blue
}
\begin{document}
\title{Compiling our First Program}
\author{Learn C With Babbo}
\date{}
\maketitle
\section*{The Direction of This Course}
In the community of programmers, one of the most common mechanisms for introducing a language, especially to students
of that language, is with a ``Hello World" program. The program is extremely simple: it just outputs ``Hello World!".
In this directory, I've included a C source file entitled ``helloworld.c". Now that we are actually going to start
compiling and running programs, most lesson directories from now on will have a C source file, which contains a sample
program and \textbf{comments} describing C programming concepts and what the program does. Comments are segments in 
source code that are meant to be read by humans and have no impact on what the program does. They usually provide
clarifications to other programs and descriptions on how the program should be used. I will frequently litter C source
files I provide with comments: these comments should be considered a crucial component of the course, and indeed
from now on, less will be explained in PDF files and more will be explained in comments in the lesson source code.
Sometimes, no PDF will be provided, and all the information will be contained in the source code comments, and other
times, I will provide both. In this case, unless otherwise noted, read the PDF first and then the source code. \\

Additionally, many lessons will now contain a PDF file called ``Exercises.pdf". This PDF will contain conceptual and 
programming assignments for you to complete. While the information needed to learn C is contained in the comments and 
PDF files, it is highly recommended that you complete these assignments: I can seldom recall instances in which I 
effectively learned programming concepts without hands-on experience. \\ \\

\section*{Hello World!}
Before you can run the ``Hello World" program you'll need to get it onto your computer, either by downloading it or 
copying its contents into a file editor and saving it. Once you have done so, navigate with your terminal to the 
directory containing ``helloworld.c", and enter \\ \\
\verb|gcc helloworld.c -o helloworld| \\ \\
on Mac/Linux or \\ \\
\verb|gcc helloworld.c -o helloworld.exe| \\ \\
on Windows.
This \textbf{compiles} the program. To run it, enter \\ \\
\verb|./helloworld| \\ \\
on Mac/Linux or \\ \\
\verb|helloworld| \\ \\
on Windows. Hopefully, you see the \textbf{output} ``Hello World!" in your console. As I said, the first command
compiled the program: \verb|gcc| is the C compiler command, where \verb|helloworld.c| is the name of the file 
(source code) we are compiling. \verb|-o| is what is called an \textbf{option}. By convention, options are denoted
with a single hyphen when they consist of one letter. Here, \verb|o| is short for ``output". When we provide the flag
\verb|-o|, we are also expected to provide a file name to save the resulting program or \textbf{executable} as. 
The file name we provided was \verb|helloworld| on Mac/Linux and \verb|helloworld.exe| on Windows: this file name is
\textbf{an argument to the option} \verb|-o|. Note that the suffix \verb|.exe| is not required on Windows, but it is a 
convention on Windows to have \verb|.exe| as a suffix for executable files. Indeed, the following lines will have the
same effect on Windows: \\ \\
\verb|gcc helloworld.c -o helloworld| \\ \\
\verb|helloworld| \\ \\
Note that on Windows we can omit the suffix \verb|.exe| when running an executable ending in \verb|.exe|, as we did
previously. Entering \verb|helloworld.exe| is the same as entering \verb|helloworld| in that case. \\
Putting all together, \\ \\
\verb|gcc helloworld.c -o helloworld| \\ \\
means ``compile \verb|helloworld.c| and output (\verb|-o|) the program into a file called \verb|helloworld|". Note that
the \verb|-o| option, like most command options, is optional (it is actually not uncommon for some commands to require
that an option be given). If we omit it, \verb|gcc| will instead output the program to an executable called \verb|a|
on Mac/Linux or \verb|a.exe| on Windows, which is obviously not very informative, so you should always specify the
\verb|-o| option. Note that \verb|gcc| will replace any existing file in your output path with your program: thus,
if we already have a file called \verb|helloworld| and we run \\ \\
\verb|gcc helloworld.c -o existing_file| \\ \\
and there is already a file called \verb|existing_file| in your directory, it will replace it with your program.
Usually, this is a convenient way for us recompile a program after making changes to it, but obviously you shouldn't
do anything silly like giving the path to that new book you're writing for the \verb|-o| argument :). \\ \\

After we compiled our program, we ran it using its name. In Mac/Linux (Bash), we prefixed the program name with \\ \\
\verb|./| \\ \\
so we ended up entering \\ \\
\verb|./helloworld| \\ \\
Recall from lesson 0.3 
(\href{https://github.com/partiallyderived/Learn-C-With-Babbo/blob/master/Chapter0_Prerequisites/0.3_Terminals/Terminals.pdf}{Terminals})
that \verb|.| represents our current directory, so we are basically saying 
``run the executable \verb|helloworld| in the directory I am in". In contrast, you do not need the \verb|./| prefix if
you are in a different directory: for example, you could also execute ``helloworld" from the parent directory as 
follows: \\ \\
\verb|cd ..| \\ \\
\verb|<name of directory containing helloworld>/helloworld| \\ \\
However, if you tried to omit the \verb|./| prefix in the directory containing \verb|helloworld|, you would almost
certainly get an error message such as \\ \\
\verb|helloworld: command not found| \\ \\
This is because Bash, unlike Windows, does not search the current directory for executables when you have only provided a 
\textbf{base name}, or the rightmost component of a path, components being separated by forward slashes on UNIX and
back slashes on Windows. On Windows, if there is an executable in the current directory, you may use just its basename
as we did to run \verb|helloworld.exe| earlier. Of course, you can run executables from other directories if you want: \\ \\
\verb|chdir ..| \\ \\
\verb|<name of directory containing helloworld.exe>\helloworld| \\ \\

\section*{Compile and Run}
From now on, I will frequently use the phrase ``compile and run" to mean compile, and then run a program. For example,
if on a future PDF I say ``compile and run example.c", I mean to first compile \verb|example.c| using \\ \\
\verb|gcc example.c -o <name of executable>| \\ \\
where unless otherwise stated, you are free to name the executable whatever you like. Generally, when I compile a 
single source file, I simply name the executable with its extension omitted or replaced, for instance ``example" on Mac/Linux or ``example.exe" on Windows.
But again, you are free to choose. After compiling it, you should run the executable using \\ \\
\verb|./<name of executable (e.g., example)>| \\ \\
on Mac/Linux or just \\ \\
\verb|<name of executable (e.g., example.exe)>| \\ \\
on Windows. \\ \\

Now that you have compiled and run a program, answer the questions in ``Exercises.pdf" to check your understanding
of this material.

\end{document}