\documentclass{article}

\begin{document}
\title{The git And gcc Commands}
\author{Learn C With Babbo}
\date{}
\maketitle

\section*{}
Now that we've become loosely acquainted with Bash, we need to install two commands for it that will be essential this
course. These commands are \verb|git| and \verb|gcc|. Refer to your operating system's subsection for how to install
them.  Note that there are no sections provided that describe how to install \verb|git| and \verb|gcc| on 
Windows systems that cannot install the Ubuntu subsystem, or Linux machines which
are not Ubuntu, but there are plenty of resources available online for those interested. Note that it is likely if you
are running Linux that you already have \verb|git| and \verb|gcc| installed. \\

\subsection*{Windows and Ubuntu}
To install \verb|git| and \verb|gcc| on Windows running the Ubuntu subsystem or an Ubuntu machine, 
enter the following lines in Bash, which will prompt you for your UNIX user password: \\ \\
\verb|sudo apt-get update| \\
\verb|sudo apt-get install git| \\
\verb|sudo apt-get install gcc| \\

After you've run these lines, enter \verb|git| and you should see some text describing how to use \verb|git|. 
If instead you see something similar to ``command not found", something went wrong with your installation. \\ 
Likewise, enter \verb|gcc|, and you should see something like ``fatal error: no input files" if the installation 
succeeded. If it says command not found or something similar, \verb|gcc| was not properly installed. \\

You won't need to know \verb|sudo| or \verb|apt-get| to learn C, but I'll give some brief context for those interested.
The command \verb|sudo| means ``do as superuser" (su = superuser, do = do). It is basically synonymous with something like
``Run as Adminstrator" that you may have seen on Windows: whatever follows \verb|sudo| is executed as a privileged/super
user named ``root", and running as root allows you to perform tasks that require special permissions, such as those for
administrators, like installing software. \\

The command \verb|apt-get| is a software management utility for Ubuntu 
(not Bash in general) that lets you install software directly from Bash. The first command \verb|sudo apt-get update|
updates the list of available software packages, so that if new software became available since the last time you ran
\verb|update|, you will be able to install it. The next two commands are self-explanatory: they utilize
\verb|apt-get install| to install \verb|git| and \verb|gcc|.
\subsection*{Mac}
\verb|git| and \verb|gcc| are provided on a Mac system through the Xcode development tools. To install them, go to the 
APP Store and search for Xcode and install it. After installation is complete, open a new Terminal (do not use a 
Terminal that was already open) and enter \verb|git|. You should see text describing how to use \verb|git|. Similarly,
enter \verb|gcc|, after which you should see something similar to ``fatal: no input files." If either of these commands
output something like ``command not found", the installation was probably unsuccessful and you may need to 
troubleshoot online.

\section*{The git Command}
\verb|git| is a \textbf{version control} system, which allow for the storage, update, and management of software 
projects. We won't be using it much in this course, except to get the latest updates and additions to the lessons. 
Nonetheless, it is an extremely useful tool that is used in the majority of software development companies to organize
code and other assets produced by teams. \\ 

For this course, we will first use \verb|git| to ``clone" (download) the storage space for this course, which is called
a \textbf{repository}. To do so, open bash and pay attention to your working directory, since you will be navigating to
a directory we will create here to complete the lessons. Enter the following command: \\ \\
\verb|git clone https://github.com/partiallyderived/Learn-C-With-Babbo.git| \\
This will download the course materials into a directory called ``learn-c-with-babbo" on your computer. Next, enter the
directory with \verb|cd learn-c-with-babbo| and list the directory contents with \verb|ls|, and you should see the
various lessons for the course. Whenever new lessons are added, navigate to this directory and enter
\verb|git pull| to download the latest changes. \\

To actually view the course materials, it will probably be more convenient for you to navigate to this folder with
your computer's file system interface. On Windows, you can probably find it easily by pressing the Windows key and
typing ``learn-c-with-babbo", where on a Mac you could navigate to it with Finder. The next lesson will have
recommendations for text-editors that are suitable for programming in (though for what it's worth, you can code entirely
on the command line if you want to).

\section*{The gcc Command}
\verb|gcc| is a compiler that converts C code into machine code. I will go into more detail on \verb|gcc| when
we start programming (L1.1).
\end{document}
