\documentclass{article}
\usepackage{inconsolata}
\usepackage{indentfirst}
\usepackage{listings}
\usepackage{xcolor}
\usepackage{textcomp}
\usepackage[T1]{fontenc}

\definecolor{cbackground}{rgb}{0.9, 0.9, 0.9}
\definecolor{cstring}{rgb}{0, 0.5, 0}

\lstset{language=C, upquote=true}
\lstdefinestyle{CStyle}{
    backgroundcolor=\color{cbackground},
    keywordstyle=\color{blue},
    stringstyle=\color{cstring},
    basicstyle=\ttfamily,
    breakatwhitespace=false,         
    breaklines=true,                 
    captionpos=b,                    
    keepspaces=true,              
    numbersep=5pt,                  
    showspaces=false,                
    showstringspaces=false,
    showtabs=false,                  
    tabsize=2,
    language=C
}


\begin{document}
\title{Hints Chapter 3}
\author{Learn C With Babbo}
\date{}
\maketitle
You can use integer division and the modulo (\%) operator to get the total of the largest denomination and the remaining
amount. For example, \verb|736 / 100| gives 7, for 7 dollars, and \verb|736 % 100| gives 36, for the remaining 36 cents.
But the modulo operator and integer division will only work on integers, and the dollar amount is entered as a
floating-point number. Can you think of a way to convert it to an integer using the example I just gave?


\end{document}