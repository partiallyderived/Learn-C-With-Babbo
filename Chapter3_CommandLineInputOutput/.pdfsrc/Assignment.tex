\documentclass{article}
\usepackage{inconsolata}
\usepackage{indentfirst}
\usepackage{listings}
\usepackage{xcolor}
\usepackage{textcomp}
\usepackage[T1]{fontenc}

\definecolor{cbackground}{rgb}{0.9, 0.9, 0.9}
\definecolor{cstring}{rgb}{0, 0.5, 0}

\lstset{language=C, upquote=true}
\lstdefinestyle{CStyle}{
    backgroundcolor=\color{cbackground},
    keywordstyle=\color{blue},
    stringstyle=\color{cstring},
    basicstyle=\ttfamily,
    breakatwhitespace=false,         
    breaklines=true,                 
    captionpos=b,                    
    keepspaces=true,              
    numbersep=5pt,                  
    showspaces=false,                
    showstringspaces=false,
    showtabs=false,                  
    tabsize=2,
    language=C
}


\begin{document}
\title{Programming Assignment for Chapter 3}
\author{Learn C With Babbo}
\date{}
\maketitle
Congratulations, you now know enough about C to create some simple programs that may be useful, which means it's time 
for your first programming assignment! 

This programming assignment has two parts. The first part is simple: you will first prompt the user for their
first and last names, and print their full name back to them, using prompts like the following:
\begin{verbatim}
Enter your first name: 
Enter your last name:
Hi <first name> <last name>!
\end{verbatim}

Once the user has entered their first and last name, your task is to is to give them their change for some item they
are buying. To do so, first ask the user to input the cost of the item in dollars, with the decimals being the number
of cents, and then how much they are paying for the item:
\begin{verbatim}
Enter the item's cost: 
Enter the amount paid:
You are owed $<paid - cost> in change!
\end{verbatim}
The dollar sign should be omitted. You may assume \verb|paid >= cost|. Then, you are to output the number of dollars, 
quarters, dimes, nickels, and pennies the user is owed, using the largest denominations possible before going to smaller
denominations. For example:
\begin{verbatim}
Enter the item's cost: 3.14
Enter the amount paid: 10.50
You are owed $7.36 in change!
That would be:
7 dollars,
1 quarters,
1 dimes,
0 nickels,
and 1 pennies!
\end{verbatim}
For now, you can leave the "s" on even if the quantity is one: once we learn conditionals, you will see how to
handle this case specially.

I have included "Hints.pdf" for assistance in completing this assignment. I recommend trying to complete the assignment
without hints first, and then using the Hints.pdf if you are having trouble. It is crucial that you are able to complete
this assignment correctly before you continue in this course.
\end{document}